\chapter*{Abstract}

In this work, we study the load balancing problem comparing the Push-Pull Sum protocol proposed in Nugroho et al. \cite{nugroho2023PushPullSumDataAg} to the Single Proposal Load Balancing protocol proposed in Dinitz et al. \cite{dinitz2022localDealAgreementloadBalancing}. In undirected graphs, nodes can transfer loads to their neighbors, aiming to achieve a balanced state in the network. These loads may represent computational tasks related to CPU usage, memory utilization, or even internet traffic. Balancing loads and therefore challenging over- and underloading helps improve the efficacy of distributed systems and prevent system and performance errors from occurring. In cloud computing, load balancing algorithms are crucial for improving response times, ensuring system stability, and therefore contributing to customer satisfaction.

To provide a comprehensive analysis, we implement the aforementioned load balancing algorithms and evaluate their performance through simulations. Simulations are conducted using the PeerSim simulation tool, where we compare the progression of the mean squared error across multiple computation rounds. The simulations are performed on various topologies with different characteristics to identify the limitations and strengths of each algorithm.

\todo{One or two sentences regarding results and outcome. Sneak peak. One Sentence regarding self-stabilization}

\iffalse
    In this work, we study load balancing algorithms in peer-to-peer (P2P) networks with different topologies. We consider the push-pull sum algorithm as well as the deal-agreement-based algorithm. The latter algorithms have been described by Yefim Dinitz et al. in "Local Deal-Agreement Algorithms for Load Balancing in Dynamic General Graphs". \cite{dinitz2022localDealAgreementloadBalancing}

    To depict a comprehensive analysis, we implement the mentioned load balancing algorithms and evaluate their outcomes using simulations. Simulations are conducted using the PeerSim simulation tool. The push-pull-sum algorithm is implemented as proposed in Saptadi Nugroho et al., "Adding Pull to Push Sum for Approximate Data Aggregation". \cite{nugroho2023PushPullSumDataAg}.

    The study aims not only to compare the simulations of the implemented algorithms but also to investigate limits and opportunities. The simulations depict the behaviour of the Mean Squared Error over 50 rounds each for different network sizes.
\fi