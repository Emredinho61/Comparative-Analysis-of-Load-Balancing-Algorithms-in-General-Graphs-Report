\chapter{Conclusion}\label{chap:conclusion}
When we looked at the results from our simulations, we noticed a few important things. The Single-Proposal Deal-Agreement-Based protocol was able to completely balance the network for two types of graphs (complete graph and torus graph with a network size of $2^{4}$). This means it could reduce the error to zero. In larger networks, the Push-Pull Sum protocol came very close to this goal, reducing the error to almost $1e-19$.  

The Single-Proposal Deal-Agreement-Based protocol is an \textit{anytime} algorithm. This means that, during the balancing process, the state of the network will never worsen from one computation round to another. In the simulations of the star graph, it looks like as if the Push-Pull Sum protocol does not have this property, although this may be due to inaccuracies with double representations.

The Push-Pull Sum protocol performs very well in networks with high connectivity, such as in complete graphs, star graphs, lollipop graphs, and ring of cliques. In these graphs a node has a larger selection of neighbors when looking for a trading partner. Some nodes balance themselves out more quickly if they are selected more often by trading partners in a round. However, a network structure with lower connectivity (torus, closed chain graph) is more beneficial to the Single-Proposal Deal-Agreement-Based protocol.

All in all, we have observed that the respective protocols draw their strengths from the characteristics of the topologies in which they operate. We have analysed the reasons why the protocols in these networks behave the way they do.