\chapter{Conclusion}\label{chap:conclusion}
When we looked at the results from our simulations, we noticed a few important things. he DAB protocol was able to completely balance the network for two types of graphs (Complete Graph and Star Graph) with a network size of 2424. This means it could reduce the error to zero. In larger networks, the PPS protocol came very close to this goal, reducing the error to almost $1e-19$.  

The Push-Pull Sum protocol is no "anytime" algorithm. This means that sometimes, during the balancing process, the error can temporarily increase before it starts to decrease again. This can be a problem if you need a steady improvement. On the other hand, the DAB algorithm does have the anytime property—it consistently reduces the error over time without any unexpected increases.

The PPS protocol performs very well in networks with high connectivity, such as in Complete Graphs, Star Graphs, Lollipop Graphs, and Ring of Cliques. In these graphs a node has a larger selection of neighbors when looking for a trading partner. Some nodes balance themselves out more quickly if they are selected more often than trading partners in a round. However, a network structure with lower connectivity (torus, closed chain graph) is more beneficial to the Single-Proposal-Deal-Agreement-Based protocol.

All in all, we have observed that the respective protocols draw their strengths from the characteristics of the topologies in which they operate. We have analysed the reasons why the protocols in these networks behave the way they do. We have seen that the protocols have different characteristics and behave differently in different network sizes.