\chapter{Introduction}\label{chap:introduction}
In the course of digitalisation, the efficient handling of data loads is becoming increasingly important. Load balancing algorithms are utilized to combat over- and underloading of nodes in a undirected graph. The main goal here, is to achieve a balanced state of the network, so to make the loads of each node approach that of the average of the network. 

In this report two different algorithms are studied across different topologies and network sizes. One of which is the Push-Pull Sum protocol proposed in paper \cite{nugroho2023PushPullSumDataAg} by Nugorho et al.. The Push-Pull Sum protocol is a combination of the Push protocol proposed in \cite{kempe2003gossipbasedComp} and the Pull-Sum protocol. The Push-Pull Sum protocol is a randomized protocol. The Push-Pull Sum protocol is compared to the Single-Proposal Load Balancing algorithm proposed in the paper \cite{dinitz2022localDealAgreementloadBalancing}. Unlike the Push-Pull Sum protocol the Single-Proposal Load Balancing protocol is a deterministic load balancing algorithm. The Single-Proposal Load Balancing algorithm is a deal-agreement-based algorithm, which means that every load transfer is based on a negotiation between the neighbors on how large the load transfer is going to be. As a measure for arriving at the goal state the Mean Squared Error is utilized to give a statement of how much the current state of the network derived from the ideal, balanced state.

Load balancing is utilized in distributed systems such as cloud computing, where it is important to efficiently allocate ressources in order to reduce power consumption and improve execution times \cite{Aghdashi2022NovelDynamicLoadBalancing}. Especially in customer-oriented areas, self-stabilisation is an important property that load balancing algorithms should definitely have. For the reason that an algorithm should always be able to balance the network regardless of what state the network is in. This is exactly what the property of self-stabilisation ensures.

\textit{Contribution.} In this report a comprehensive analysis is provided comparing the two load balancing algorithms regarding Mean Squared Error Values per round. The simulations are conducted for different network sizes and topologies to test the strengths and drawbacks of each algorithm under different circumstances aiming to verify the properties and performance of the algorithms of the papers \cite{nugroho2023PushPullSumDataAg} and \cite{dinitz2022localDealAgreementloadBalancing}. 

\section{Problem Overview}
Consider a undirected graph of computers. Each computer in this undirected graph fulfils different tasks and therefore has a computation load. Each computer has the ability to share to or receive tasks from other computers which are connected via edges to them, in order to reduce or increase their computational load. In principle, it is possible to send any load over the edges, while ensuring that the state of the network does not worsen and become more "unbalanced". Furthermore, care should be taken not to carry out too many load transfers per node and round, as these are heavy operations. The aspect of self-stabilisation must also be taken into account. In terms of load balancing, self-stabilizing algorithms, are algorithms that eventually ensure a balanced state of the network, no matter in which state the network has been.
\section{Document Structure}
\todo{once document is finished, give a broad overview over each section/subsection}
