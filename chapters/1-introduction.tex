\chapter{Introduction}\label{chap:introduction}
In the course of digitalisation, the efficient handling of data loads has become increasingly critical. Load balancing algorithms are utilized to combat over- and underloading of nodes in a undirected graph. The primary objective of these algorithms is to achieve a balanced state within the network, where the load on each node converges toward the average load of the entire network. 

This report examines two distinct load balancing algorithms across various network topologies and sizes. The first algorithm studied is the Push-Pull Sum protocol, introduced by Nugroho et al. \cite{nugroho2023PushPullSumDataAg}. The Push-Pull Sum protocol is a protocol, that combines the Push protocol \cite{kempe2003gossipbasedComp} with the Pull-Sum protocol. The Push-Pull Sum protocol is a randomized protocol. The Push-Pull Sum protocol is compared to the Single Proposal Load Balancing protocol proposed by Dinitz et al. \cite{dinitz2022localDealAgreementloadBalancing}. Unlike the Push-Pull Sum protocol the Single Proposal Load Balancing protocol is a deterministic load balancing algorithm. The Single Proposal Load Balancing algorithm is a deal-agreement-based algorithm, which means that every load transfer is based on a negotiation between the neighbors on how large the load transfer is going to be. To evaluate the performance of these algorithms, the Mean Squared Error (MSE) is employed as a metric, quantifying how far the network's current state deviates from the ideal, balanced state. A network is balanced if the MSE equals 0.

Load balancing is utilized in distributed systems such as cloud computing, where it is important to efficiently allocate ressources in order to reduce power consumption and improve execution times \cite{Aghdashi2022NovelDynamicLoadBalancing}. Particularly in customer-oriented domains, self-stabilization is a desirable property for load balancing algorithms. This property ensures that an algorithm can always achieve a balanced state, regardless of the initial state the network is in.

\textit{Contribution.} This report provides a comprehensive analysis of two distinct load balancing algorithms by comparing their performance in terms of Mean Squared Error (MSE) per round. Analysis are provided for simulations conducted across various network sizes and topologies, allowing us to get an insight into the strengths and limitations of each algorithm under different conditions. The simulations may be used to validate the properties and performance claims made in the original papers by Nugroho et al. \cite{nugroho2023PushPullSumDataAg} and Dinitz et al. \cite{dinitz2022localDealAgreementloadBalancing}.

\section{Problem Overview}
Consider a undirected graph of computers. Each computer in this undirected graph is assigned tasks and therefore has a computation load. To manage and balance these loads, computers can transfer tasks to or receive tasks from their neighboring computers, connected by edges in the graph.
In practice, while it is theoretically possible to transfer any amount of load across the edges, it is important to ensure that transfers do not contribute to imbalance in the network. Additionally, it is important to minimize the number of load transfers per node per round, as these operations are resource-intensive and can lead to inefficiencies if overused.

\section{Document Structure}
In the \hyperref[chap:background]{Algorithms and Concepts} chapter, we introduce the two load balancing algorithms as described in the referenced papers. This chapter also includes a detailed description of the simulation setup, under which the simulations are conducted (eg. number of rounds, simulation techniques...).

The tools and methodologies used for the simulations are discussed in the \hyperref[chap:simulations]{Simulations} chapter. Here, we introduce the chosen simulation tool, and provide implementation details aiming to contribute to a better understanding of the simulation results.

In the chapter \hyperref[chap:results]{Result and Evaluation of Simulations}, we present the simulation outcomes, including detailed descriptions of the topologies and an analysis of the results. This chapter concludes with a tabular summary highlighting the performance of each algorithm under different conditions.

Finally, the \hyperref[chap:appendix]{Appendix} contains additional simulation results, including those using powers of 10 instead of powers of 2, which are not covered in the main part of the report.
